\section{Astor \-- GenProg}
%
\begin{frame}{}
	\begin{center}
		\huge{Astor \-- GenProg}
	\end{center}
\end{frame}
%
\begin{frame}{Astor}
	\begin{itemizes}{1em}
		\item Wendet eins der drei Modi an
		\begin{itemizes}{0.2em}
			\item jGenProg2
			\item jKali
			\item jMutRepair
		\end{itemizes}
		\item Unser Fokus wurde auf GenProg gelegt
	\end{itemizes}
\end{frame}
%
\begin{frame}{Kali}
	\begin{itemizes}{1em}
		\item Zielt auf schwache Testsuits
		\item Vorgehen bei der "Reperatur":
		\begin{itemizes}{0.2em}
			\item löschen von Zeilen
			\item überspringen von Zeilen
		\end{itemizes}
	\end{itemizes}
\end{frame}
%
\begin{frame}{MutRepair}
	\begin{itemizes}{1em}
		\item Mutiert die Konditionen von if-Statements
		\item Hat drei Arten der Mutation:
		\begin{itemizes}{0.2em}
			\item Relations Operationen
			\item Logische Operationen
			\item Negation
		\end{itemizes}
	\end{itemizes}
\end{frame}
%
\begin{frame}{GenProg}
	\begin{itemizes}{1em}
		\item Idee: Reparatur durch Evolution
		\begin{itemizes}{1em}
			\item Nutzung von generischer Programmierung
			\item gezielte, zufällige Mutation
		\end{itemizes}
	\end{itemizes}
\end{frame}
%
\begin{frame}{Vorgehen}
	\begin{itemizes}{1em}
		\item Fehlerlokalisierung
		\begin{itemizes}{0.5em}
			\item Erstellung eines abstrakten Syntax Baums
			\item Durchlaufen der Testfälle
			\item Fehlerbestimmung anhand von Pfaden mit negativen Testfällen
		\end{itemizes}
		\item Patch-Generierung
		\begin{itemizes}{0.5em}
			\item Mutation von Crossovervarianten
			\item Solange, bis ein optimaler Kandidat gefunden wurde
		\end{itemizes}
		\item Validierung
		\begin{itemizes}{0.5em}
			\item Prüfen des Testsuits
		\end{itemizes}
	\end{itemizes}
\end{frame}
%

%
%
%
%
%
%
%
%
%
%
%
%
%
%
%
%
%
%
